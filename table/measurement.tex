\begin{table}[h]
	\centering \footnotesize
	\caption{Indicators of Fiscal Policy Conservatism}
	\label{tab:policies} 
	\begin{tabular}{p{5.5cm}P{3cm}P{4.5cm}} \hline
		\textbf{Policy}                          & \textbf{Availabiliy \newline (number of years)} & \textbf{Do Higher or Lower Values Imply Conservatism?} \\
		\hline
		&&\\ \textit{Tax policy} &&\\
		Income tax (pct.)                        & 29     &    Lower       \\
		Property tax (per mille)                      & 29    &    Lower        \\
		Commercial real estate tax (per mille) & 14    &    Lower               \\ \hline
	
		&&\\ \textit{Spending policy}  &&\\
		Spending pr. capita (DKK)               & 29    &    Lower        \\
		Spending pr. pupil in school (DKK)       & 7     &    Lower     \\ \hline
		
		&&\\\textit{Organization of public service delivery}  &&\\
 		Public Employees (pr. 1,000 citizens)	 & 9	  &	   Lower	     \\
 		Privately operated services  (pct.) &   14  &    Higher     \\
 		Purchases with a private supplier  (pct.)      & 14    &    Higher     \\ \hline
 	
 		&&\\ \textit{Co-payment for public services} &&\\   
		Average cost of day care (DKK)                  & 16    &    Higher     \\
		Price of relief stay (DKK)				 & 7	  &	   Higher	 \\
		Food delivery for the  elderly (DKK) & 7   &    Higher     \\
		Stay in nursing home (DKK)              & 7     &    Higher     \\ \hline
	
		&&\\ \textit{Extent of Public Services} &&\\ 
		Public housing (pct.)                    & 14     &    Lower               \\
		Class size in public schools	         & 14    &    Lower       \\
		\hline \hline
		\multicolumn{3}{p{13cm}}{Notes: There was a change in how certain parts of social spending was measured in 1994. We adjust for this in our analysis, subtracting the average difference between '78--'93 and '94--'05 from the spending variable after '94.}
		\end{tabular}
\end{table} 