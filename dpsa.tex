\documentclass[a4paper,11pt]{article}


%calling packages
\usepackage[english]{babel}
\usepackage[utf8]{inputenc}
\usepackage{amsmath}
\usepackage{graphicx}
\usepackage[left=1.4in,right=1.4in,top=1.3in,bottom=1.3in]{geometry}
\usepackage{setspace}
\usepackage[round]{natbib}
\usepackage{epstopdf}
\usepackage{soul}
\usepackage{lmodern}
\usepackage{caption}
\usepackage{hyperref}
\usepackage{longtable}
\usepackage{amssymb}
\usepackage{fancyhdr}
\usepackage{array}
\newcolumntype{P}[1]{>{\centering\arraybackslash}p{#1}}
%fonts
\usepackage{tgpagella}
\setcounter{secnumdepth}{0}
%chanfing font of table headers
\captionsetup[figure]{labelfont=bf}
\captionsetup[table]{labelfont=bf}

%changing header
\pagestyle{fancy}
\fancyhf{}
\rhead{\thepage}
\renewcommand{\headrulewidth}{0pt}
\renewcommand{\footrulewidth}{0pt}
\renewcommand*\footnoterule{}
\let\svfootnoterule\footnoterule
\renewcommand\footnoterule{\vspace{0.2in}\svfootnoterule}

%set spacing

\renewcommand{\sfdefault}{phv}

\onehalfspacing
\usepackage{titlesec}

\titleformat*{\section}{\Large\sffamily}
\titleformat*{\subsection}{\large\sffamily}
\titlespacing*\section{0pt}{24pt plus 4pt minus 2pt}{4pt plus 2pt minus 2pt}
\titlespacing*\subsection{0pt}{20pt plus 4pt minus 2pt}{4pt plus 2pt minus 2pt}


%changing title settings
\makeatletter
\renewcommand{\maketitle}{
	\begin{flushleft}
		
		\onehalfspacing
		
		\@title
		
		\lineskip .5em
		\normalfont{\normalsize{\@author}}
\end{flushleft}}
\makeatother


\newcommand{\beginsupplement}{%
	\setcounter{table}{0}
	\renewcommand{\thetable}{S.\arabic{table}}%
	\setcounter{figure}{0}
	\renewcommand{\thefigure}{S.\arabic{figure}}%
}


%title
\title{\bigskip \bigskip \sffamily \LARGE The Extent and Antecedents of Dynamic Responsiveness in Municipal Government}

%author
\author{\bigskip Benjamin Carl Egerod\footnote{Graduate Student, e-mail: \texttt{xx@ifs.ku.dk}.} \qquad Martin Vinæs Larsen\footnote{Research Assistant, e-mail: \texttt{vinaes@gmail.com}.} \\ \textit{Department of Political Science, University of Copenhagen}} %e-mail



\begin{document}
	
	
	
	
	
	
	\begin{footnotesize} \noindent \today \end{footnotesize} %date
	
	\vspace{0.7in}
	
	\maketitle
	
	\bigskip
	
	\begin{quotation} %abstract
		\singlespace
		\small \noindent \emph{Abstract:} Local governments supposedly empower citizens, giving them the ability to shape their local community according to their beliefs. In this article, we explore whether this democratic promise is met in the Danish municipalities. In particular, we look at whether the policy implemented by local politicians actually respond to changes in the public mood and whether this type of responsiveness is more common in some municipalities. To do this, we develop a measure of municipal fiscal policy conservatism. Based on 16 policy indicators, the measure covers all primary Danish municipalities from 1978 to 2006. We link this data to a measure of ideological sentiment in the Danish municipalities (i.e., support for left-wing parties). using a generalized difference-in-difference setup, we find strong evidence for dynamic responsiveness: if public opinion in a municipality changes then public policy responds. Finally, we identify an effect of single party control of the city council on the level of responsiveness. We identify this effect using a close elections regression discontinuity design, comparing the responsiveness of city councils where the largest party narrowly won a majority of the seats with city councils where the largest party narrowly lost.
	\end{quotation}
	
	
	\thispagestyle{empty} %removing page number from page one
	
	
\section{Introduction} %first section goes here

\section{Empirical Context: Danish Municipalities}	

\section{An Annual Measure of Municipal Fiscal Policy Conservatism}

\subsection{Some Descriptive Statistics}

\section{Dynamic Responsiveness}

\subsection{Measuring Municipal Liberalism}

\section{The Effect of Governing Alone}

\subsection{A Regression Discontinuity}

\subsection{Results}
	
	
\end{document}
